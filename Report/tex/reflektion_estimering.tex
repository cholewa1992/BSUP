\documentclass[../report.tex]{subfiles}

\begin{document}

\section{Refleksion}

\paragraph{Brug af planl�gningspoker}

I starten af projektet var medlemmers estimater meget forskellige. Dette var grundet at gruppens medlemmer ikke havde samme forst�else af hvad opgaven indebar. Pokeren havde derfor den effekt, at der ikke blot blev n�et konsensus om opgavens st�rrelse, men ogs� om indhold, hvilket f�rte til at gruppens medlemmer havde bedre forst�else af projektets f�lles retning. Senere i forl�bet var estimaterne mere ens grundet gruppens bedre forst�else af projektet som helhel og de enkelte delopgaver. Brugen af poker var derfor en succes i target-projektet.

\paragraph{Brug af PERT}
Fordelen ved denne model er, at man tager hensyn til forskellige scenarier og PERT kan med fordel kombineres med planl�gningspoker, hvor gruppens medlemmer i hver budrunde estimerer hvert scenarie og dets sandsynlighed.


I target-projektet blev PERT brugt til at estimere enkelte opgaver med succes, men konklusionen p� denne metode er at det bliver at PERT og Scrum ikke fungere s� godt sammen. Tidsforbruget p� at bruge PERT ved opgave estimering i Scrum planl�gningen opvejes ikke af det bedre estimat. 
\end{document}