\documentclass[../report.tex]{subfiles}


\begin{document}

\graphicspath{{img/}{../img/}}

\section{Delphi}

Delphi er en estimeringsmetode der bygger p� antagelsen af en gruppes antagelse er bedre end en individuel antagelse. Metoden er udviklet i 1950'erne og er oprindeligt til estimering af krigssituation hvor en gruppe eksperter anonymt i estimeringers runder gav deres estimat af fjendtlige angreb p� interessepunkter. Efter hver runde blev der snakket om de forskellige estimater hvorefter en ny runde begyndte indtil der var opn�et consensus om et estimat. Det vigtig ved denne metode er anonymitet for at undg� kognitiv forankring;  menneskers tendens til at g� med det f�rste tilg�ngelige information n�r en beslutning skal tr�ffes.


\paragraph{Hvad er planl�gnings poker?}

Planl�gnings poker, eller scrum poker, bygger p� delphi metoden. Grupper laver bud runder hvor hver medlem kommer med sit anonyme bud p� en opgaves st�rrelse. Dette forg�r med et s�t spillekort med v�rdierne fra fibonacci talr�kke. Hvert medlem l�gger sit kort med forsiden nedad og alle kort vendes s� p� en gang. Resultatet diskuteres og en ny runde begynder indtil der er opn�et enighed.

Grunden til at bruge fibonacci talr�kke i steder for at estimere i timer er at det ofte forvirrer at estimerer i timer, hvor en relativ enhed er nemmere af forholde sig til. Ved adskillelse har man ogs� bedre mulighed for at m�le velocity, antal point pr timer, hvilket er relevant hvis man �nsker at m�le p� en gruppes produktivitet.

\paragraph{Planl�gnings poker i target-projektet} I target-projekt er planl�gnings blevet brugt ved sprint planl�gning til at estimere sprint backlog items. I starten af projektet var estimaterne meget forskellige grundet at gruppens individer ikke havde samme forst�else af hvad opgaven indebar. Pokeren havde derfor den effekt af der ikke blot var consensus om opgavens st�rrelse, men ogs� om indhold.

\todo NOGLE KONKRETE TAL P�!


\end{document}