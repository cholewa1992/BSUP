\documentclass[../report.tex]{subfiles}
\begin{document}
\graphicspath{{img/}{../img/}}

\section{Reflektioner om procesmodeller}

\label{sec:Vandfaldsmodellen}


I target-projektet er der, som tidligere beskrevet, arbejdet ud fra en udgave af den iterative processmodel Scrum. At der er arbejdet iterativt med target-projektet har betydet at der l�bende har v�ret mulighed for at kunne tilpasse systemdesignet ud fra nye krav, samt at arbejde p� skiftende funktionalitet undervejs. \\

I forl�bet med target-projektet har det for target-gruppen v�ret en klar opgave at f� defineret alle krav i samarbejdet med de studerende fra SMU s� hurtigt som muligt. Undervejs fors�gte target-gruppen at undg� at tilf�je flere krav, s� der var en veldefineret kravspecifikation, der fortalte om hvilken funtionalitet, der skulle implementeres for at projektet kunne verificeres og afsluttes. P� baggrund af dette kan man argumentere for, at der er arbejdet ud fra en procesmodel der l�ner sig meget op af vandfaldsmodellen med hensyn til den indelende fase i projektet, hvor der kun blev arbejdet p� at definere krav til systemet. De efterf�lgende faser m� dog siges at have v�ret mere iterative, da der er fortaget forskellige designvalg sidel�bende med implementering af den n�dvendige funktionalitet. \\

Hvis der i target-projektet var blevet arbejdet fuldt ud efter vandfaldsmodellen, ville de forskellige faser i projektet have v�ret tydeligt specificeret som ogs� modellen angiver. I dette tilf�lde ville processen med at specificere krav ikke v�re meget anderledes, da denne fase var endeligt besluttet inden target-gruppen p�begyndte systemdesign og implementation af den n�dvendige funktionalitet. Hvis man havde valgt at arbejde ethundrede procent efter vandfaldsmodellen skulle design af systemet besluttes og l�gges fast inden p�begyndelse af implementation. P� denne m�de ville man kunne opn� en mere fast struktur, som netop f�lger analogien i vandfaldsmodellen. Dette kunne i target-projektet give et mere overskueligt projekt, hvor det tydeligt fremg�r hvor meget af det valgte system, der allerede er implementeret, og hvor meget der mangler. Det kr�ver dog ogs� en stor viden inden for arkitekturen af det projekt man �nsker at designe, da denne som sagt skal l�gges fast p� forh�nd og efterf�lgende ikke m� eller skal �ndres. Desuden ville der kr�ves mere af produkt- og aktivitetsnedbrydningen, da det ellers ville v�re sv�rt for target-gruppen at estimere aktiviteternes varighed.


\end{document}