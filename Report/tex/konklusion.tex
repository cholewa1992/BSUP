\documentclass[../report.tex]{subfiles}
\begin{document}

%\textit{Hvordan kan man planl�gge og udf�re et eksamensprojekt s� som target-projektet?} \\

%\noindent Dette sp�rgsm�l er i l�bet af rapporten blevet besvaret, og der vil her samles op p� delkonklusionerne fra tidligere kapitler. 

I denne rapport er der blevet unders�gt forskellige planl�gningsmetoder, estimeringsmetoder og procesmodeller i forhold til target-projektet for at svare p� problemformuleringen. I denne unders�geslse er der blevet lavet en analyse af target-projektet, som er blevet sammenlignet med relevante alternativer ud fra teori. Ud fra dette har vi kunnet drage f�lgende konklusioner.

Ud fra analysen kan det konkluderes, at en passende procesmodel kan tildele et projekt god struktur, der hj�lper med at holde projektets m�l. For at v�lge en passende procesmodel er det n�dvendigt at analysere projektets karakteristika og ud fra denne v�lge den rette procesmodel. Det konkluderes at agile procesmodeller egner sig godt til projekter som target-projektet, hvor viden om l�sningsdom�net l�bende udvikles, og specifikt Scrum da korte sprints tvinger arbejdsresultater gennem hele processen.

Det kan ogs� konkluderes at god planl�gning af et projekt som target-projektet, kan g�res med et Gantt-chart. Gantt-chartet var godt til den agile procesmodel, fordi en delvis nedbrydning af projektet var tilstr�kkelig. Dette gjorde at der kunne holdes overblik over de forskellige hovedaktiviteter i projektet, og hvorn�r aktiviteterne skulle udf�res, uden omfattende forh�ndsanalyse og dokumentation. 

Analysen viser, at der med fordel kan benyttes planl�gningspoker til sprintplanl�gning i projekter som target-projektet. Ud fra analysen konkluderes det at et s�dant estimeringsv�rkt�j hj�lper med at holde sprint backloggen i den rigtige st�rrelse og at v�rkt�jet skaber bedre f�lles forst�else for indholdet af projektets delopgaver og dermed projektet som helhed.

For at kunne udarbejde analysen, er der blevet valgt forskellige metoder og ud fra diskussionen kan det konkluderes, at metoderne de fleste steder har v�ret tilstr�kkelige. Dog har forkert metodik medf�rt at analysen enkelte steder var utilstr�kkelig. Det m� derfor konkluderes at metoderne i h�jere grad skulle have inddraget target-projektgruppen og i h�jere grad have trukket p� praktiske erfaringer i stedet for teoretisk opstillede modeller.

P� trods af dette er der ud fra den samlede analyse, vist hvordan man kan planl�gge og udf�re et eksamensprojekt som target-projektet. 


\end{document}