\documentclass[../report.tex]{subfiles}
\begin{document}
\section{Use Case Points}

Use Case Points(UCP)\footnote{\url{{http://www.cs.cmu.edu/~jhm/Readings/Cohn\%20-\%20Estimating\%20with\%20Use\%20Case\%20Points\_v2\%2012-24-50-761.pdf}} er en metode til at estimere st�rrelsen af software projekter. Metoden bygger p� en r�kke forskellige m�lbare tal som vil blive angivet med f�lgende forkortelser:
\begin{itemize}
\item UUCP - Unadjusted Use Case Points, den r� score for alle use cases.
\item UAP - Unadjusted Actor Points, den r� score for alle actors.
\item TCF - Technical Complexity Factor, en faktor der siger noget om systemet tekniske kompleksitet.
\item ECF - Environmental Complexity Factor, en faktor der siger noget om kompleksiteten af det milj� systemet bliver udviklet i.
\item UCP - Use Case Points, den samlede score for estimeringen.
\end{itemize}
N�r man har beregnet de fire f�rste kan man beregne UCP ud fra f�lgende formel:\[UCP = (UAP + UUCP) \cdot TCF \cdot ECF\]

\subsubsection{UUCP}
For at beregne UUCP tildeler man hver enkelt use case en score baseret p� antallet af transaktioner i den enkelte use case. Id�en er at et h�jt antal transaktioner medf�rer en h�j kompleksitet. Per definition af UCP metoden er en use case med 1-3 transaktioner af lav kompleksitet og tildeles 5 UUCP, 4-7 transaktioner betyder at use casen har en mellem kompleksitet og tildeles 10 UUCP og alt over 7 transaktioner bliver vurderet til en h�j kompleksitet hvor use casen bliver tildelt 15 UUCP. Den samlede UUCP for systemet er ganske simpelt summen af UUCPerne fra hver use case.

\subsection{UAP}
For at beregne UAP tildeler man hver enkelt actor en score baseret p� om det er:
\begin{itemize}
\item Et eksternt system der skal interagere med systemet vha. en veldefineret API. - 1 point
\item Et eksternt system der skal interagere med systemet vha. standard protokoller. - 2 point
\item Et menneske der skal interagere med systemet vha. en grafisk brugerflade. - 3 point
\end{itemize}
Den samlede UAP score er summen af de tildelte point.

\subsection{TCF}
N�r man beregner TCF kigger man p� 13 forskellige tekniske aspekter\footnote{For en fuld liste over hvilke faktorer der kigges p�, se appendix.} af systemet der skal udvikles, bl.a. hvor nemt det skal v�re at vedligeholde, om det skal v�re distribueret og hvor stor v�gt der skal l�gges p� sikkerhed. Hver af disse faktorer har en medf�dt sv�rhedsgrad, som fx 2,0 for et distribueret system. Denne sv�rhedsgrad ganges med en faktor mellem 0 og 5 baseret p� hvor vigtigt det individuelle tekniske aspekt er for systemet, hvor 0 betyder at aspektet er irrelevant for systemet, og 5 betyder at aspektet har h�j relevans for systemet. N�r man har v�gtet alle 13 aspekter kan man summere den score man f�r(TF) og beregne TCF p� f�lgende m�de: \[TCF = (0,6 + \frac{TF}{100})\]
\subsection{ECF}
ECF beregnes p� en m�de tilsvarende TCF. Her kigger man dog p� milj�m�ssige faktorer i stedet for tekniske. Dette inkluderer faktorer som hvor bekendte udviklerne er med UML, hvor stabile kravene til systemet er og hvor 
\subsection{UCP i target-projektet}
I target-projektet er der defineret 11 use cases og de kan tildeles UUCP som vist p� tabel \ref{tab:UUCP}. Summen af UUCP  er alts� 80.
\begin{table}[!h] 
\centering
\begin{tabular}{|c|c|c|}
\hline Use Case & Antal af transaktioner & UUCP  \\ 
\hline UC-01 & 4 & 10  \\ 
\hline UC-02 & 6 & 10 \\
\hline UC-03 & 6 & 10 \\
\hline UC-04 & 3 & 5 \\
\hline UC-05 & 4 & 10 \\
\hline UC-06 & 3 & 5 \\
\hline UC-07 & 4 & 10 \\
\hline UC-08 & 3 & 5 \\
\hline UC-09 & 3 & 5 \\
\hline UC-10 & 3 & 5 \\
\hline UC-11 & 3 & 5 \\
\hline
\end{tabular}
\caption{} \label{tab:UUCP}
\end{table}

Ligeledes kan man tildele en samlet UAP p� 9 da der er 3 forskellige menneskelige actors.

\end{document}