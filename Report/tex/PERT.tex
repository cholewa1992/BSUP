\documentclass[../report.tex]{subfiles}

\begin{document}

\section{Program Evaluation and Review Technique} 
PERT, Program Evaluation and Review Technique, er en estimeringsmetode, hvor forskellige udfald medregnes. Der tages h�jde for sandsynligheden for at opgaven kan l�ses hurtigere og der medregnes en risiko for at opgaven tr�kker ud. Estimatet p� en opgave kan s�ledes udregnes med f�lgende formel: \\

\begin{align*}
T_E = \frac{aO + bM + cP + \cdots + zN}{a+b+c+ \cdots + z}
\end{align*}\\


hvor,\\
O er et optimistisk estimat og a er dets sandsynlighed \\
M er det mest realistiske estimat og b er dets sandsynlighed \\
P er et  pessimistiske  estimat og c er dets sandsynlighed \\ \\
Modellen kan udvides ved inds�ttelse af flere scenarier og deres sandsynlighed i $zN$ og $z$.\\


%N�r man laver de tre forskellige estimater er det vigtigt, at man analyserer hvad kriterierne er for at scenariet opfyldes. Hvis et projekt, hvis alt g�r godt er f�rdig p� 2 m�neder, men 3 m�neder hvis man medregner sm� forhindringer forhindringer og 6 m�neder hvis der m�des st�re problemer skal man kigge p� hvor sandsynligt de tre forskellige senarier er. Nedenfor ses et udregnet eksemplet hvor realistiske estimat er dobbelt s� sandsynligt som det optimistiske som er dobbelt s� sandsynligt som det pessimistiske.

\todo QA note fra Thomas: F�rste linje herunder virker uklar. Jeg er usikker p�, om du mener at opgaven man estimerer skal v�re klart defineret eller om vi snakker om noget helt andet? \\

N�r man laver de tre forskellige estimater er det vigtigt, at man analyserer hvad kriterierne er for at scenariet opfyldes. For at opstille et eksempel, kan der t�nkes p� en aktivitet, der optimistisk set kan f�rdigg�res p� 2 dage, men hvis der opleves mindre forhindringer, hvilket er realistisk at antage, vil tage 3 dage og hvis der m�des st�rre problemer vil tage 6 dage. Det vurderes at det realistiske estimat er n�sten dobbelt s� sandsynligt som det optimistiske, der igen er dobbelt s� sandsynligt som det pessimistiske. Nedenfor ses det udregnede estimat p� eksemplet.

 \begin{align*}
 T_E = \frac{30\%\cdot 2 + 55\%\cdot 3 +15\%\cdot 6}{30\%+55\%+15\%} = 3.15
 \end{align*}\\

Fordelen ved denne model er, at man tager hensyn til forskellige scenarier og PERT kan med fordel kombineres med planl�gningspoker, hvor gruppens medlemmer i hver budrunde estimerer hvert scenarie og dets sandsynlighed.

\end{document}
