\documentclass[../report.tex]{subfiles}
\begin{document}
\graphicspath{{img/}{../img/}}

\section{Reflektioner om procesmodeller}

\label{sec:Vandfaldsmodellen}

Hvis der i target-projektet var blevet arbejdet fuldt ud efter vandfaldsmodellen, ville de forskellige faser i projektet have v�ret tydeligt specificeret som ogs� modellen angiver. I dette tilf�lde ville processen med at specificere krav ikke v�re meget anderledes, da denne fase var endeligt besluttet inden target-gruppen p�begyndte systemdesign og implementation af den n�dvendige funktionalitet. Hvis man havde valgt at arbejde ethundrede procent efter vandfaldsmodellen skulle design af systemet besluttes og l�gges fast inden p�begyndelse af implementation. P� denne m�de ville man kunne opn� en mere fast struktur, som netop f�lger analogien i vandfaldsmodellen. Dette kunne i target-projektet give et mere overskueligt projekt, hvor det tydeligt fremg�r hvor meget af det valgte system, der allerede er implementeret, og hvor meget der mangler. Det kr�ver dog ogs� en stor viden inden for arkitekturen af det projekt man �nsker at designe, da denne som sagt skal l�gges fast p� forh�nd og efterf�lgende ikke m� eller skal �ndres. Desuden ville der kr�ves mere af produkt- og aktivitetsnedbrydningen, da det ellers ville v�re sv�rt for target-gruppen at estimere aktiviteternes varighed. 

%I target-projektet er der, som tidligere beskrevet, arbejdet ud fra en simpel udgave af den iterative processmodel Scrum. At der er arbejdet iterativt med target-projektet har betydet at der l�bende har v�ret mulighed for at kunne tilpasse systemdesignet ud fra nye krav, samt at arbejde p� skiftende funktionalitet undervejs. \\

For target-projektgruppen har det v�ret en tydelig opgave at f� defineret alle krav s� tidligt i forl�bet som muligt. Dette medf�rte at de lukkede for tilgangen af nye krav undervejs i projektet, og gjorde at de havde en tydelig definition af hvilken funktionalitet der skulle implementeres for at projektet kunne verificeres og afsluttes. P� baggrund af dette kan man argumentere for, at der er arbejdet ud fra en procesmodel der l�ner sig meget op af vandfaldsmodellen med hensyn til den indledende fase i projektet, hvor der kun blev arbejdet p� at definere krav til systemet. \\

Hvis target-projektgruppen havde valgt at arbejde stringent efter vandfaldsmodellen ville de efterf�lgende faser have v�ret mindre iterative. Her kan man ikke tillade sig f.eks. at lave designvalg sidel�bende med implementering, som deres valg af agil procesmodel ellers l�gger op til. En mere tydelig struktur med arbejdet i de forskellige faser ville fremg� ved brug af vandfaldsmodellen, da hver enkelt fase ville v�re veldokumenteret inden afslutning. Ved at arbejde ud fra denne model kr�ves der stor viden om arkitekturen af det projekt man �nsker at designe, og dette kan v�re problematisk i et eksamensprojekt...


\end{document}