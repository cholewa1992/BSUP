\documentclass[../report.tex]{subfiles}

\begin{document}

\section{Refleksion}

\paragraph{Brug af planl�gningspoker}

I starten af projektet var medlemmers estimater meget forskellige. Dette var p� grund af, at gruppens medlemmer ikke havde samme forst�else af hvad opgaven indebar. Pokeren havde derfor den effekt, at der ikke blot blev n�et konsensus om opgavens st�rrelse, men ogs� om indhold, hvilket f�rte til at gruppens medlemmer havde bedre forst�else af projektets f�lles retning. Senere i forl�bet var estimaterne mere ens grundet gruppens bedre forst�else af projektet som helhel og de enkelte delopgaver. Brugen af poker var derfor en succes i target-projektet.

\paragraph{Brug af PERT}
Da vi anvendte PERT metoden p� target-projektet opn�ede vi et resultat der var lidt t�ttere p� det faktiske tidsforbrug end det estimat udviklerne kom frem til vha. planl�gningspoker. Der bruges dog ekstra tid p� at estimere med PERT metoden frem for planl�gningspoker og derfor fremst�r PERT ikke som et bedre alternativ til planl�gningspoker i target-projektet.

\paragraph{Brug af UCP}
UCP metoden til at estimere st�rrelsen af et IT-projekt beror sig p� Use Cases, hvilke typisk bliver brugt, n�r kravene til systemet er klart defineret. I denne forstand passer metoden udem�rket til et eksamensprojekt som target-projektet, da der bliver stillet nogle rimeligt pr�cise krav til systemet i s�danne projekter. Derimod er estimatet p� 1640 mandetimer, som metoden kom frem til, langt h�jere end det faktiske tidsforbrug af udviklerne i target-projektet. Dette skyldes nok i nogen grad, at metoden tager h�jde for faktorer, som ikke n�dvendigvis g�r sig g�ldende i et mindre eksamensprojekt, s� som fuld dokumentation og testing af systemet. 

\paragraph{Estimering i eksamensprojekter}
Selvom gruppen af udviklere i target-projektet havde nogle gode erfaringer med planl�gningspoker, var disse erfaringer ikke direkte forbundet med selve estimatet, men mere at de forskellige estimeringer fra gruppens medlemmer lagde op til diskussion om hvad de forskellige opgaver indebar. Derudover viser det store overestimat som UCP metoden gav, at denne metode ikke er specielt brugbar til estimering af eksamensprojekter. 

Overordnet set er det derfor vores vurdering at det ikke giver nogen s�rlig gevinst i forhold til udf�relsen af et eksamensprojekt, s� som target-projektet, at estimere dets st�rrelse.
\end{document}