\documentclass[../report.tex]{subfiles}
\begin{document}
Use Case Points(UCP) er en metode til at estimere st�rrelsen af software projekter. Metoden bygger p� en r�kke forskellige m�lbare tal som jeg vil angive med f�lgende forkortelser:
\begin{itemize}
\item UUCP - Unadjusted Use Case Points, den r� score for alle use cases.
\item UAP - Unadjusted Actor Points, den r� score for alle actors.
\item TCF - Technical Complexity Factor, en faktor der siger noget om systemet tekniske kompleksitet.
\item ECF - Environmental Complexity Factor, en faktor der siger noget om kompleksiteten af det milj� systemet bliver udviklet i.
\item UCP - Use Case Points, den samlede score for estimeringen.
\end{itemize}
N�r man har beregnet de fire f�rste kan man beregne UCP ud fra f�lgende formel:\[UCP = (UAP + UUCP)*TCF*ECF\]

\subsubsection{UUCP}
For at beregne UUCP tildeler man hver enkelt use case en score baseret p� antallet af transaktioner i den enkelte use case. Id�en er at et h�jt antal transaktioner medf�rer en h�j kompleksitet. Per definition af UCP metoden er en use case med 1-3 transaktioner af lav kompleksitet og tildeles 5 UUCP, 4-7 transaktioner betyder at use casen har en mellem kompleksitet og tildeles 10 UUCP og alt over 7 transaktioner bliver vurderet til en h�j kompleksitet hvor use casen bliver tildelt 15 UUCP. Den samlede UUCP for systemet er simpelthen summen af UUCPerne fra hver use case.

\subsection{UAP}
For at beregne UAP tildeler man hver enkelt actor en score baseret p� om det er:
\begin{itemize}
\item Et veldefineret API - 1 point
\item Et svagt defineret API - 2 point
\item Et menneske - 3 point
\end{itemize}
Den samlede UAP score er summen af de tildelte point.

\subsection{TCF}

\subsection{ECF}

\subsection{Vores m�lprojekt}
I vores m�lprojekt har vi defineret 11 use cases og de kan tildeles UUCP som f�lger:
\begin{table}[!h] 
\begin{tabular}{|c|c|c|}
\hline Use Case & Antal af transaktioner & UUCP  \\ 
\hline UC-01 & 4 & 10  \\ 
\hline UC-02 & 6 & 10 \\
\hline UC-03 & 6 & 10 \\
\hline UC-04 & 3 & 5 \\
\hline UC-05 & 4 & 10 \\
\hline UC-06 & 3 & 5 \\
\hline UC-07 & 4 & 10 \\
\hline UC-08 & 3 & 5 \\
\hline UC-09 & 3 & 5 \\
\hline UC-10 & 3 & 5 \\
\hline UC-11 & 3 & 5 \\
\hline
\end{tabular}
\caption{} \label{tab:UUCP}
\end{table}
Summen af UUCP  er alts� 80.

Ligeledes kan man tildele en samlet UAP p� 9 da der er 3 forskellige menneskelige actors.

\end{document}