\documentclass[../report.tex]{subfiles}
\begin{document}
\graphicspath{{img/}{../img/}}

\section{Reflektioner om procesmodeller}

\label{sec:Vandfaldsmodellen}

%\paragraph{Brug af Scrum}

\paragraph{Brug af Vandfaldsmodellen}

Hvis target-projektgruppen havde valgt at arbejde med vandfaldsmodellen som procesmodel istedet for Scrum, havde projektet udformet sig en smule anderledes. Target-projektgruppen har lagt tydelig v�gt p� at definere deres kravspecifikation som det f�rste i forl�bet. Efter en aftalt kravspecifikation med krav fra egen og SMU'ernes side var p� plads, har de ikke tilf�jet nye krav i resten af forl�bet. P� baggrund af dette kan man argumentere for, at der er arbejdet ud fra en procesmodel der l�ner sig meget op af vandfaldsmodellen med hensyn til den indledende fase i projektet, hvor der kun blev arbejdet p� at definere krav til systemet. De efterf�lgende faser (design og implementation) har dog vist sig at v�re mere iterative for target-projektgruppen. \\

Da target-projektgruppen har udtalt at de har lavet designvalg sidel�bende med implementeringen af funktionalitet, viser det at de har anvendt det iterative workflow, som Scrum stiller til r�dighed. De har l�bende taget designbeslutninger som var n�dvendige for at f�rdigg�re det p�g�ldende sprint. Fordelen ved at bruge den agile procesmodel er at man i hele processen kan �ndre p� designvalg og specifikke implementeringer uden at bryde modellens analogi. Target-projektgruppen har l�bende kunne �ndre i interfaces, benytte andre design-patterns eller udskifte hele komponenter i systemet, ved blot at tage denne med som opgave i n�ste sprint. Dette er ikke muligt i vandfaldsmodellen. Her kr�ver det at gruppen p� forh�nd har fastsl�et alle designbeslutninger i design-fasen, og der ville derfor ikke v�re mulighed for lave �ndringer l�bende. Modellen er derfor p� sin vis ufleksibel, men dette kan ogs� v�re en fordel, da man i target-projektet tydeligere ville kunne f�lge fremskridt og milep�le, som naturligt genereres ved brug af vandfaldsmodellen. \\

At target-projektgruppen har valgt Scrum som deres procesmodel kan muligvis baseres p� det faktum at deres projekt er et eksamensprojekt. De befinder sig alts� i en l�ringssituation hvor en agil procesmodel vil v�re en fordel. Man kunne forestille sig at de kunne f� ny viden om alternativer til arkitekturen, som de har brug for at implementere, hvilket er muligt at g�re l�bende i Scrum. At kunne arbejde med individuelle adskildte faser som i vandfaldsmodellen, kr�ver p� forh�nd stor viden om arkitekturen for lignende systemer, samt viden pr�cist om hvilke alternativer man har til r�dighed.


\end{document}