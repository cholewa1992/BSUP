\documentclass{article}

\usepackage[latin1]{inputenc}
\usepackage[english]{babel}
\usepackage[T1]{fontenc}
\usepackage{amsmath,amssymb}
\usepackage{fancyhdr}
\usepackage{hyperref}
\usepackage{graphicx}
\usepackage{tabularx}
\usepackage{float}


\fancyfoot[OC]{\textit{\thepage}}

\title{Samarbejdsaftale BSUP}
\date{\today}

\author{Asbj\o rn Fjelbro Steffensen (afjs@itu.dk)\\ Thomas Stoy Dragsb\ae k (thst@itu.dk)\\ Nicki Hjorth J\o rgensen (nhjo@itu.dk)\\ Jacob Cholewa (jbec@itu.dk)\\ Jakob Merrild (jmer@itu.dk)\\ Mathias Pedersen (mkin@itu.dk)}

\begin{document}

\maketitle
\newpage

\section{Kontaktoplysninger}
Thomas Dragsb\ae k, thst@itu.dk, 20553025 \\
Nicki Hjorth J\o rgensen, nhjo@itu.dk, 53343780 \\
Asbjørn Steffensen, afjs@itu.dk, \\
Jacob Cholewa, jbec@itu.dk, 
\section{Gruppens m\o der}
Mandag efter undervisning og eventuelt weekender, hvis n\o dvendigt.

\section{Arbejdsform}
Foruden m\o derne om mandagen vil vi fors\o ge at uddele opgaverne s\aa \ meget som muligt.  

\includegraphics{WBS.png}
\begin{figure}
\includesvg{WBS.svg}
\end{figure}


\subsection{Definition of done}
Man har selv l\ae st teksten igennem og tjekket om korrektur samt indhold tilfredsstillende. Derefter skal det reviewes og godkendes af gruppen i et review møde. S\aa fremt det ikke godkendes af gruppen, skal teksten omskrives med hensyntagen til gruppens feedback.

\section{Ambitionsniveau}
Vi vil lave en rapport, der bruger de ting vi har l\ae rt i undervisningen.

\section{Straf}
Et medlem kan smides ud af gruppen, hvis der har v\ae ret to kollektive advarsler uden den \o nskede effekt. Kollektive advarsler kan kun gives, hvis gruppens medlemmer minus 2 finder det n\o dvendigt. Eksempler p\aa \ ting, der kan udl\o se en advarsel: Ikke overholdte deadlines, manglende fremm\o de.


\end{document}