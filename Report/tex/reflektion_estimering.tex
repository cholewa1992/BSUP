\documentclass[../report.tex]{subfiles}

\begin{document}

\section{Refleksion over estimeringsmetoder \textsc{\textcolor{light-gray}{ - Jacob Cholewa \& Jakob Merrild}}}

\paragraph{Brug af planl�gningspoker}
I starten af projektet var medlemmers estimater meget forskellige. Dette var p� grund af, at gruppens medlemmer ikke havde samme forst�else af hvad opgaven indebar. Pokeren havde derfor den effekt, at der ikke blot blev n�et konsensus om opgavens st�rrelse, men ogs� om indhold, hvilket f�rte til at gruppens medlemmer havde bedre forst�else af projektets f�lles retning. Senere i forl�bet var estimaterne mere ens grundet gruppens bedre forst�else af de enkelte delopgaver og projektet som helhed.

\paragraph{Brug af PERT}
Ved at anvendte PERT metoden p� en aktivitet fra target-projektet, opn�edes et resultat der var lidt t�ttere p� det faktiske tidsforbrug end det estimat target-projektgruppen kom frem til vha. planl�gningspoker. Der bruges dog ekstra tid p� at estimere med PERT metoden frem for planl�gningspoker, idet der skal opn�s enighed om 3 estimater og deres sandsynligheder.

\paragraph{Brug af UCP}
UCP metoden til at estimere st�rrelsen af et IT-projekt beror sig p� Use Cases, hvilke typisk bliver brugt, n�r kravene til systemet er klart defineret. I denne forstand passer metoden udem�rket til et eksamensprojekt som target-projektet, da der bliver stillet nogle rimeligt pr�cise krav til systemet i s�danne projekter. Target-projektgruppen har kun taget tid p� deres aktiviteter i 4 ud af de 16 uger som projektet forl�b over. Dette g�r det sv�rt at sammenligne deres faktiske tidsforbrug med det resultat som UCP metoden kom frem til. Baseret p� target-projektgruppens fornemmelse virker estimatet p� 1180 timer dog til at v�re h�jere end det faktiske tidsforbrug. Dette skyldes nok i nogen grad, at metoden tager h�jde for faktorer, som ikke n�dvendigvis g�r sig g�ldende i et mindre eksamensprojekt, s� som fuld dokumentation og testing af systemet. 

\paragraph{Estimering i eksamensprojekter}
Selvom gruppen af udviklere i target-projektet havde nogle gode erfaringer med planl�gningspoker, var disse erfaringer ikke direkte forbundet med selve estimatet, men mere at de forskellige estimeringer fra gruppens medlemmer lagde op til diskussion om hvad de forskellige opgaver indebar. PERT metoden viste sig at v�re lidt t�ttere p� virkeligheden, men tog til geng�ld l�ngere tid end planl�gningspoker. Derudover viser det store overestimat som UCP metoden gav, at denne metode ikke er specielt brugbar til estimering af eksamensprojekter. Derfor foretr�kkes planl�gningspokeren, n�r der skal estimeres i eksamensprojekter.
 
\end{document}