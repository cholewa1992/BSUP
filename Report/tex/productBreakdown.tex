\documentclass[../report.tex]{subfiles}
\begin{document}

\graphicspath{{img/}{../img/}}

\section{Product Breakdown}
I target-projektet brugte de et Product Breakdown Diagram til at hj�lpe dem med deres planl�gning.


Et Product Breakdown Diagram laves ved at se p� et helt projekt og finde ud af hvilke dele det best�r af. Man inddeler alts� projektet i produkter der, n�r de bliver sat sammen, vil udg�re et f�rdigt projekt. Hvert produkt kan s� deles ind i mindre produkter, ligesom man g�r med hele projektet \cite[p.~59-61]{hughescotterell09}. Produkterne vil p� den m�de danne et hierarki, hvor det �verste produkt vil v�re hele projektet, og de nederste produkter vil v�re detaljerede produkter, som er mere h�ndgribelige.

Ved at lave et Product Breakdown Diagram f�r man et bedre overblik over, hvad der skal laves for at projektet bliver f�rdigt. Senere i projektet kan man ogs� bruge det til at holde styr p�, om alle dele af projektet er blevet lavet.

I target-projektet blev der lavet et Product Breakdown Diagram med relativt f� produkter (se figur \ref{fig:Product Breakdown Diagram}). Det er med andre ord ikke blevet brudt ned i meget detaljerede produkter. Selve Product Breakdown Diagrammet blev prim�rt brugt i starten af projektet til, at f� et overblik over hvad der skulle laves, men det blev ogs� brugt som basis for at lave et Activity Breakdown Diagram. 


\begin{figure}[H]
\hspace{-2.1cm}
\includegraphics[ scale=0.9]{ProductBreakdownDiagram.pdf}
\caption{Product Breakdown Diagram fra target-projektet}
\label{fig:Product Breakdown Diagram}
\end{figure}



\end{document}