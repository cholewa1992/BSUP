\documentclass[../report.tex]{subfiles}


\begin{document}

\graphicspath{{img/}{../img/}}

\section{Delphi}

Delphi er en estimeringsmetode der bygger p� antagelsen af, at en gruppes estimat er bedre end den individuels estimat. Metoden er udviklet i 1950'erne og er oprindeligt brugt til estimering af krigssituation. En gruppe eksperter laver en budrunde og byder anonymt ind med deres estimat af sandsynlighed for fjendtlige angreb p� interessepunkter. Alle estimaterne bliver offentlige for gruppen p� samme tid. Efter hver runde blev der snakket om de forskellige estimater hvorefter en ny runde begyndte indtil der var opn�et consensus om et estimat. 

Det vigtig ved denne metode er at anonymitet og den samtidige offentligg�relse eliminerer kognitiv forankring;  menneskers tendens til at g� med det f�rste tilg�ngelige information n�r en beslutning skal tr�ffes.


\paragraph{Hvad er planl�gnings poker?}

Planl�gnings poker, eller scrum poker, bygger p� delphi metoden. Grupper laver bud runder hvor hvert medlem kommer med sit anonyme bud p� en opgaves st�rrelse. Pokeren bliver udf�rt med et s�t spillekort med v�rdierne fra fibonacci talr�kke og hvert medlem l�gger sit kort med forsiden nedad. Alle kort vendes s� p� en gang. Resultatet diskuteres og en ny runde begynder indtil der er opn�et enighed.

Grunden til at bruge fibonacci talr�kke i steder for at estimere i timer er, at det ofte forvirrer at estimerer i timer hvorimod en relativ enhed er nemmere af forholde sig til. Ved adskillelse har man ogs� bedre mulighed for at m�le velocity, antal point pr timer, hvilket er relevant hvis man �nsker at m�le p� en gruppes produktivitet.

\paragraph{Planl�gnings poker i target-projektet.} I target-projekt er pokeren blevet brugt ved sprint planl�gning til at estimere sprint backloggen. I starten af projektet var estimaterne meget forskellige grundet at gruppens individer ikke havde samme forst�else af hvad opgaven indebar. Pokeren havde derfor den effekt af der ikke blot var consensus om opgavens st�rrelse, men ogs� om indhold, hvilket f�rte til at gruppens individer havde bedre forst�else af projektet f�lles retning. Senere i forl�bet var estimaterne mere ens grundet gruppes bedre forst�else af projektet som helhel og de enkelte delopgaver. Brugen af poker var derfor en succes i target-projketet.


\end{document}