\documentclass[../report.tex]{subfiles}


\begin{document}

\graphicspath{{img/}{../img/}}

\section{Delphi}
\label{sec:Delphi}

Delphi er en estimeringsmetode der bygger p� antagelsen af, at en gruppes estimat er bedre end den individuelles estimat. Metoden er udviklet i 1950'erne og er oprindeligt brugt til estimering af krigssituationer\footnote{The Modified Delphi Technique - A Rotational Modification (http://scholar.lib.vt.edu/ejournals/JVTE/v15n2/custer.html)}. Det foregik p� den m�de, at en gruppe eksperter i en budrunde anonymt b�d ind med deres estimat af sandsynligheden for fjendtlige angreb p� forskellige interessepunkter. Alle estimaterne blev offentlige for gruppen p� samme tid. Efter hver runde blev der snakket om de forskellige estimater, hvorefter en ny runde begyndte indtil der var opn�et konsensus om et estimat. 

Det vigtige ved denne metode er at anonymiteten og den samtidige offentligg�relse eliminerer kognitiv forankring; menneskers tendens til at g� med det f�rste tilg�ngelige information, n�r en beslutning skal tr�ffes.


\paragraph{Hvad er planl�gnings poker?}

Planl�gnings poker, eller Scrum poker, bygger p� Delphi metoden. En gruppe laver budrunder, hvor hvert medlem kommer med sit anonyme bud p� en opgaves st�rrelse. Pokeren bliver udf�rt med et s�t spillekort med v�rdierne fra Fibonaccis talr�kke og hvert medlem l�gger sit kort med forsiden nedad. Alle kort vendes s� p� samme tid, hvorefter resultatet diskuteres og en ny runde begynder. Processen gentages indtil der er opn�et enighed.

Grunden til at bruge Fibonaccis talr�kke i stedet for at estimere i timer er, at det ofte forvirrer at estimere i timer, hvorimod en relativ enhed er nemmere at forholde sig til. Ved adskillelse har man ogs� bedre mulighed for at m�le \textit{velocity}, antal point pr timer, hvilket er relevant, hvis man �nsker at m�le p� en gruppes produktivitet.


\paragraph{Planl�gnings poker i target-projektet.} I target-projekt er pokeren blevet brugt ved sprintplanl�gning til at estimere sprint backloggen. Det foregik som beskrevet ovenfor, ved at have budrunder med anonyme bud, for derefter at diskutere buddene og gentage processen indtil der blev opn�et konsensus.
\end{document}