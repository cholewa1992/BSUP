\documentclass[../report.tex]{subfiles}

\begin{document}
\graphicspath{{img/}{../img/}}


\section{Gantt planl�gning}
Gantt-chartet er en udbyggelse af den klassiske tidsplan. Det er lavet for at give et bedre overblik over de forskellige planlagte opgaver i forhold til hinanden. Hvor en klassisk tidsplan er fokuseret p� tidsperioderne, er et Gantt-chart fokuseret p� arbejdsopgaverne.\\

Figur \ref{fig:Gantt} viser Gantt-chartet for target-projektet. Aktiviteterne p� diagrammet er meget abstrakte, og det kommer af target-projektgruppens valg af procesmodel. De har valgt at bruge Scrum til softwareudviklingsaktiviteterne, hvilket de p� Gantt-chartet har allokeret cirka halvdelen af projektforl�bet til. Da Scrum er en iterativ proces bliver aktiviteterne nedbrudt og valgt l�bende, og de har alts� valgt ikke bare at holde disse aktiviteter abstrakte p� Gantt-chartet, men ogs� aktiviteter f�r og efter det reelle Scrum-forl�b, som bliver beskrevet mere i afsnit \ref{sec:Scrum}.\\


\todo{Evt. noget mere om resten af aktiviteterne}

\begin{figure}[H]
\includegraphics[scale=0.45]{TargetProjectGantt.png}
\caption{Gantt diagram for target projektet}
\label{fig:Gantt}
\end{figure} 

Gantt-chartet vinder stort inden for overblik mod den klassiske tidsplan, men i n�ste kapitel kigges der p� et andet alternativ til aktivitetsplanl�gning.



\end{document}