\documentclass[../report.tex]{subfiles}

\begin{document}

\section{Reflektioner}
%Product Breakdown og Activity Breakdown reflektioner
I target-projektet blev der lavet Product og Activity Breakdown Structures, hvilket gjorde at gruppen blev tvunget til at t�nke over, hvilke produkter der skulle produceres og hvilke aktiviteter der skulle gennemf�res. Derudover var disse diagrammer ogs� grundlaget for udviklingen af Gantt-chartet.

Gruppen kunne med fordel have overvejet at lave et samlet diagram og opn�et samme effekt - her t�nkes der p� et hybridt Work Breakdown Structure, som er en kombination af de to ovenst�ende, og som laves ved f�rst at identificere hvilke produkter, der skal produceres og for hvert produkt identificere hvilke aktiviteter, der skal til for at producere produktet. 

%Gantt reflektioner
Gruppen i target-projektet havde valgt at bruge et Gantt-chart til aktivitetsplanl�gning.
En af fordelene ved at bruge et Gantt-chart er, at det giver et godt overblik over r�kkef�lgen de planglagte opgaver forventes at blive udf�rt i. Derudover passer det godt til et projekt, hvor der bruges en agil procesmodel, id�t et af grundprincipperne i det agile manifesto siger, at software der virker v�gtes h�jere end omfattende dokumentation\footnote{Manifesto for Agile Software Development - \url{http://agilemanifesto.org}}. Gantt-chartet passer netop godt til disse typer projekter, da en delvis nedbrydning af aktiviteterne er nok til at konstruere diagrammet og da estimaterne er mere l�se. Derfor er det ikke s� omfattende, og der kan alts� fokuseres p� softwaren frem for dokumentationen. Gruppen f�lte ogs�, at Gantt-chartet hjalp dem med at huske vigtige deadlines, som f.eks. fristen for aflevering af rapportudkast til review. Det gjorde, at gruppen i ugen op til deadlinen fokuserede mere p� rapporten end det havde v�ret tilf�ldet, hvis de ikke havde v�ret klar over deadlinen. 

En af ulemper ved Gantt-chartet er, at det ikke er lige s� godt som f.eks. Precedence Network til at vise afh�ngigheder p� tv�rs af aktiviteterne og man derved risikere at overse afh�ngighederne og derved pludselig st�r og er blokeret fordi en anden aktivitet ikke er udf�rt. 

%Precedence network reflektioner
Som hentydet ovenfor, ville gruppen, hvis de havde brugt et Precedence Network, have haft et bedre overblik over aktiviteternes afh�ngigheder. Derimod ville de sandsynligvis miste overblikket over de forskellige deadlines, da disse ikke kan repr�senteres i et Precedence Network. I et projekt, hvor en stor del af aktiviteterne ikke er afh�ngige af hinanden og derfor kan udf�res parallelt, giver et Precedence Network desuden ikke ret meget v�rdi. 

\end{document}