\documentclass[../report.tex]{subfiles}

\begin{document}

\section{PERT} 
PERT, Program Evaluation and Review Technique, er en estimeringsmetode, hvor forskellige udfald medregnes. Der tages h�jde for sandsynligheden for at opgaven kan l�ses hurtigere og der medregnes en risiko for at opgaven tr�kker ud. Estimatet p� en opgave kan s�ledes udregnes med f�lgende formel: \\

\begin{align*}
T_E = \frac{aO + bM + cP + \cdots + zN}{a+b+c+ \cdots + z}
\end{align*}\\


hvor,\\
O er et optimistisk estimat og a er dets sandsynlighed \\
M er det mest realistiske estimat og b er dets sandsynlighed \\
P er et  pessimistiske  estimat og c er dets sandsynlighed \\ \\
Modellen kan udvides ved inds�ttelse af flere scenarier og deres sandsynlighed i $zN$ og $z$.


\subsection{PERT i target-projektet }

For at kunne anvende PERT p� en opgave fra target-projektet, bad vi udviklerne komme med en optimistisk vurdering, den mest sandsynlige vurdering og en pessimistisk vurdering samt sandsynlighederne for disse udfald. Opgaven der skulle vuderes: Implement Details page Controller and model. Gruppen blev enige om, at det optimistisk set ville tage 1 timer. Det mest realistiske blev dog vurderet til 3 timer, men den pessimistiske vurdering l�d p� 7 timer, id�t det var en opgave, som ingen af gruppens medlemmer f�lte sig helt inde i og der derfor kunne opst� st�rre problemer, der kr�vede en del research. Gruppen mente, at der var 40\% sandsynlighed for det optimisiske vurdering, 50\% for den mest realistiske vurdering og 10\% for den pessimistiske vurdering. Med disse oplysninger kan f�lgende estimat udregnes:

 \begin{align*}
 T_E = \frac{40\%\cdot 2 + 50\%\cdot 3 +10\%\cdot 7}{20\%+50\%+30\%} = 3,0
 \end{align*}\\

Opgaven ville alts�, hvis gruppen havde brugt PERT, blive estimeret til 3 timer. Med planl�gnings poker blev opgaven estimeret til 5 point. Med en velocity (l�s mere om velocity i afsnit~\ref{sec:Delphi}) p� 0,854 ville opgaven, hvis vi omregner til timer, tage 5,85 timer. Opgaven blev aktuelt udf�rt p� 3,1 timer. \\


\end{document}
